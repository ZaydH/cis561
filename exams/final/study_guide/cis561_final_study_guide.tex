\documentclass[10pt,twocolumn]{report}

\usepackage[margin=0.3in]{geometry}
\usepackage{multicol}

\usepackage{algorithm}
\usepackage[noend]{algpseudocode}
\algdef{SE}[SUBALG]{Indent}{EndIndent}{}{\algorithmicend\ }%
\algtext*{Indent}
\algtext*{EndIndent}
\algnewcommand{\algorithmicgoto}{\textbf{go to}}%
\algnewcommand{\Goto}[1]{\algorithmicgoto~#1}%

\usepackage{enumitem}
\setitemize{itemsep=0pt}

\renewcommand{\thesection}{Lec.~\arabic{section}}
\setlength{\parindent}{0pt}

\usepackage{color}
\newcommand{\red}[1]{{\color{red} #1}}
\newcommand{\blue}[1]{{\color{blue} #1}}
\newcommand{\green}[1]{{\color{green} #1}}


\begin{document}
  \twocolumn
  
  \begin{@twocolumnfalse}
    \begin{center}
      \LARGE \textbf{CIS561 Final Exam Review}
    \end{center}
  \end{@twocolumnfalse}
  
  \section{Introduction}
  
  \textbf{Compiler Stages}:
  \begin{enumerate}
    \item \textit{Lexer/scanner}: Read text and produce token stream. FSM-based.  Tools: (F)lex
    \item Parser - Analyze grammatical structure.
    \item Abstract syntax tree \& Symbol Table
    \item Static semantics
    \item Code generation
  \end{enumerate}

  \section{Lexical Analysis}
  
  \textbf{Finite State Machine}: Consumes one character on each move and announces a pattern when matched.
  \begin{itemize}
    \item Linear time
    \item Possible States: (Non-)Accepting, Start
  \end{itemize}

  \textbf{Angelic Non-determinism}: Try all possible paths. If any ends in a final state, accept.
  
  \textbf{Limitations of Regular Expressions}: Not fully expressive for many regular languages.
  \begin{enumerate}
    \item \textit{Example}: Balanced parentheses
    \item \textit{Intuition}: Regular expressions cannot count.
  \end{enumerate}

  \textbf{Representing an NFA as a DFA}: If the NFA has $k$~states, the equivalent DFA will have \textit{at most} $2^{k}$ states.
  
  
  \textbf{Chomsky Hierarchy}: Orders languages from most to least expressive.
  \begin{itemize}
    \item \textbf{Type~0}: No restrictions
    \item \textbf{Type~1}: Context sensitive
    \item \textbf{Type~2}: Context free
    \item \textbf{Type~3}: Right linear, i.e.,~regular
  \end{itemize}
  
  \section{Parsing}
  
  \textbf{LL(*)} -- Left-to-right, leftmost derivation
  \begin{itemize}
    \item Example: ANTLR
    \item Top-down parser
    \item Rely on first and follow sets
    \item Construct right skewed trees
  \end{itemize}
  \textbf{LL1}: Also known as recursive descent parser.
  \textbf{Angelic Recursive Descent Parser}: Example shown in class that similar to an NFA, the angelic parser ``magically'' which production to select at any point in the input.

  \textbf{LR(*)}: Left-to-right, rightmost derivation
  \begin{itemize}
    \item Examples: Bison, Yacc, CUP 
    \item Bottom-Up parser: Build leaves then parents
    \item Construct left skewed trees
  \end{itemize}

  Two parser operators:
  \begin{enumerate}
    \item \textit{Peak}: Check the next character and report an error if it is invalid.
    \item \textit{Match}: Consume next token if it is the same as the expected value or throw an error.
  \end{enumerate}

  \textbf{First Set}: Set of terminals that can appear at the beginning of a specific non-terminal. \textbf{Follow Set}: Set of possible terminals can that appear after a specific non-terminal.
  
  \textbf{Example Grammar with Precedence}:
  
  \texttt{% 
  expr $\rightarrow$ expr '+' term\\
  \hbox{\hspace{30pt}}| expr '-' term\\
  \hbox{\hspace{30pt}}| term\\
  \\
  term $\rightarrow$ term '*' factor\\
  \hbox{\hspace{30pt}}| term '/' factor\\
  \hbox{\hspace{30pt}}| factor\\
  \\
  factor $\rightarrow$ '-' factor\\
  \hbox{\hspace{39pt}}| prim\\
  \\
  prim $\rightarrow$ NUMBER}
  
  \subsection*{Bottom-Up Parser Operations}
  
  \textbf{Shift}: Move over a token
  
  \textbf{Reduce}: At end of rule, reduce to non-terminal then do a GoTo.
  
  \textbf{GoTo}: Move over non-terminal. \textit{Can never have a conflict}
  
  \textbf{Dotted Item}: Represents how far along we are in the production. \textbf{Kernel}: Set of productions yielded by applying the transition operation.
  
  \subsection*{Resolving Conflicts w/ Lookahead}
  \textbf{Shift-Reduce}: Character after shift not in reduce lookahead.
  
  \textbf{Reduce/Reduce}: Intersection of all pairs reduce lookaheads empty.
  
  \section*{Languages}
  
  \noindent
  \textbf{Regular Language}: Any language that can be represented via a DFA/NFA or that can be expressed via a regular expression.
  \begin{itemize}
    \item All finite languages are inherently regular.
  \end{itemize}

\section{AST -- Why}

  \textbf{Benefits of Building an AST}
  \begin{itemize}
    \item 
  \end{itemize}

  \textbf{Type System Design Choices}
  \begin{itemize}
    \item \textit{Static versus Dynamic} -- Are variables typed or are values (respectively)?
    \item \textit{Other Object Typing} -- Functions, collections, etc. 
    \begin{itemize}
      \item What objects are \textbf{first class}, etc., can be passed to functions, returned from functions, stored in variables, etc.?
    \end{itemize}
    \item \textit{Named versus Structural} -- Can types be anonymous? Can types with different names be identical?
    \item \textit{Safety} -- What happens if the program has a bug?
    \begin{itemize}
      \item Java is safe with limited checks at runtimes (e.g., array accesses, null references)
      \item Python achieves safety by performing all checks at runtime
      \item C/C++ unsafe since the type system can be cheated. ``\textbf{Catch fire}'' semantics. By design.
    \end{itemize}
    
    \item \textit{Type Substitution} -- Inheritance, subtyping
    \item \textit{Explicit or Implicit} -- Do variables need to have their type explicitly declared or can it be inferred?
  \end{itemize}

  \textbf{Method Boundary Typing}: Simplifies inference by allowing each method to be analyzed separately.

  \textbf{Dynamic Typing} -- Allows dynamic method dispatch.
  \begin{itemize}
    \item Supported in languages such as Java, C++, and Quack
  \end{itemize}

  \textbf{Quack Static Semantic Rules}:
  \begin{itemize}
    \item No shadowing or redefinition of classes/methods
    \item Methods can be overridden but definition must be compatible.
    \item Variable must be assigned before use.
  \end{itemize}

  \textbf{Quack Class Hierarchy}
  \begin{itemize}
    \item Class hierarchy must be rooted at \texttt{Obj}
  \end{itemize}

  \textbf{Variable Initialization}
  \begin{itemize}
    \item Variables must be initialized on all paths before they are used.
  \end{itemize}

\section{Code Generation -- Branching}

  \begin{algorithm}[h]
    \caption{Generated Code for an If Statement}
    \begin{algorithmic}[1]
      \Procedure{If}{cond}
      \State  $\Call{EvalBranch}{cond, TrueLabel, FalseLabel}$
      \State  \fbox{\textbf{TrueLabel}:}
      \Indent
      \State    $\left[\left[~\textrm{TrueStatements}~\right]\right]$
      \State    \fbox{\Goto{\tt EndIf}}
      \EndIndent
      \State  \fbox{\textbf{FalseLabel}:}
      \Indent
      \State    $\left[\left[~\textrm{FalseStatements}~\right]\right]$
      \EndIndent
      \State  \fbox{\textbf{EndIf}:}
      \EndProcedure
    \end{algorithmic}
  \end{algorithm}

  \begin{algorithm}[H]
    \caption{Generated Skeleton Code for Evaluating a Boolean}
    \begin{algorithmic}[1]
      \Procedure{EvalBranch}{cond, TrueLabel, FalseLabel}
      \State $tmp \gets \Call{EvalValue}{cond}$
      \If{$tmp$} goto TrueLabel \EndIf
      \State \fbox{\Goto{FalseLabel}}
      \EndProcedure
    \end{algorithmic}
  \end{algorithm}

  \begin{algorithm}[H]
    \caption{``AND'' Short Circuit Compare}
    \begin{algorithmic}[1]
      \Function{AndEvalBr}{\textproc{And}($l,r$), TrueLabel, FalseLabel}
      \Indent
      \State \Call{EvalBranch}{$l$, \blue{HalfWayLabel}, \blue{FalseLabel}}
      \EndIndent
      \State \fbox{\textbf{HalfWayLabel}:}
      \Indent
      \State \Call{EvalBranch}{$r$, TrueLabel, FalseLabel}
      \EndIndent
      \EndFunction
    \end{algorithmic}
  \end{algorithm}

  \begin{algorithm}[H]
    \caption{``Or'' Short Circuit Compare}
    \begin{algorithmic}[1]
      \Function{OrEvalBr}{\textproc{And}($l,r$), TrueLabel, FalseLabel}
      \Indent
      \State \Call{EvalBranch}{$l$, \red{TrueLabel}, \red{HalfWayLabel}}
      \EndIndent
      \State \fbox{\textbf{HalfWayLabel}}:
      \Indent
      \State \Call{EvalBranch}{$r$, TrueLabel, FalseLabel}
      \EndIndent
      \EndFunction
    \end{algorithmic}
  \end{algorithm}

  \begin{algorithm}[H]
    \caption{``Not'' Code Generation Compare}
    \begin{algorithmic}[1]
      \Function{NotEvalBranch}{$expr$), \blue{TrueLabel}, \blue{FalseLabel}}
      \State \Call{EvalBranch}{$expr$, \red{FalseLabel}, \red{TrueLabel}}
      \EndFunction
    \end{algorithmic}
  \end{algorithm}
      
\section{Code-Improving Transformations (``Optimizations'')}
  
  \noindent
  Benefits of Optimization:
  \begin{itemize}
    \item Encourages good programing -- Reduce penalty of writing clean, well organized code
    \item Simplifies code generation -- Reduce penalty of writing simple code, which reduces bugs
    \item Make programs fast and small (least important)
  \end{itemize}

  \subsection*{Optimization Examples}

  \newcommand{\longopt}[4]{\item \textbf{#1}: #2\vspace{-6pt}\begin{itemize} \item \textit{Example}: Replace \texttt{{#3}} with \texttt{{#4}} \end{itemize}}
  \newcommand{\shortopt}[2]{\item \textbf{#1}: #2}
  \textbf{Local Optimization}: Occurs within a single basic block
  \begin{itemize}
    \setlength{\itemsep}{0pt}
    \longopt{Algebraic Identity}{(aka \textbf{Strength Reduction}) Replace computationally expensive operation with equivalent statement that is less expensive}{a = \red{x ** 2}; c = \red{a * 2};}{a = \red{x * x}; b = \red{a << 1};}
    \longopt{Constant Folding}{Replace calculations relying on constants with their fixed value equivalent}{a = 3 << 1;}{a = 6;}
    \longopt{Common Sub-expression Elimination}{Replace calculations performed multiple times with a value calculated once}{a = x * x; d = x * x;}{a = x * x; d = a;}
    \longopt{Copy Propagation}{Replace the occurrences of targets of direct assignments with their (known) values}{a = 3; b = x; c = \red{a} + \red{b};}{c = \red{3} + \red{x};}
    \shortopt{Dead Code Elimination}: Remove code that calculates and/or stores a value that is never used.
    \shortopt{Loop Fusion}{Merge two loops together. This is common when iterating over two vectors of same length}
    \shortopt{Loop Splitting}{Break large loop into multiple pieces}
    \shortopt{Peephole}{Pattern matching and replacement in a sliding window}
    \shortopt{Tiling}{Bring operations together to improve cache hit rate.}
  \end{itemize}

  \textbf{Global Optimizations}: Spans blocks within the same procedure.
  \begin{itemize}
    \item  \textbf{Code Motion}:
    \begin{itemize}
      \item \textit{Invariant Hoisting}: Move invariant code out of a loop and place before the loop.
      \item \textit{Looping Unrolling}: Duplicate loop code to reduce jumps.  Often exposes other optimizations
    \end{itemize}
  \end{itemize}

  \textbf{Inter-procedural Optimizations}: Performed across methods and may even include whole program optimization
  \begin{itemize}
    \shortopt{Inlining}{Treats a function call as a macro and copy caller could into a procedure.}
    \begin{itemize}
      \item Eliminates call overhead, e.g. stack creation/deletion
      \item May expose further optimization opportunities
    \end{itemize} 
  \end{itemize}

\section{Data Flow Analysis}

  \textbf{Def-Use Pairs}: Directed edge from where a variable is defined to where it may be used
  \begin{itemize}
    \item $<WhereDefined,WhereUsed>$
    \item Same use point may have multiple possible def points and vice versa.
  \end{itemize}

  \textbf{Live Variable}: A variable holding a value that may be needed in the future \\
  \textbf{Live Range}: Portion of the program where a variable is \textit{live}.

  \begin{itemize}
    \item \textbf{Gen}: Points in data flow graph where fact becomes \green{true}
    \begin{itemize}
      \item Analysis: Perform a ``gen'' at the \textbf{use} of a variable (i.e., \textit{become live})
    \end{itemize}
    \item \textbf{Kill}: Points in data flow graph where fact becomes \red{false}
    \begin{itemize}
      \item Analysis: Perform a ``kill'' at the \textbf{def} (store) of a variable  (i.e., get clobbered)
    \end{itemize}
  \end{itemize}

  \begin{equation}
    Live(n) = \bigcup_{m\in succ(n)}Live(m) - Kill(m) + Gen(n)
  \end{equation}
  \begin{itemize}
    \item ``-'' -- Represents set difference
    \item ``+'' -- Represents set union
  \end{itemize}

  \textbf{Bit Vector Analyses}
  \begin{itemize}
    \item \textbf{Forward Analysis}: Relies on ``\texttt{pred}''
    \item \textbf{Backward Analysis}: Relies on ``\texttt{succ}''
    \item \textbf{May Analysis}: Starts with $\emptyset$ and take union of values from neighbor
    \item \textbf{Must Analysis}: Start with everything and take intersection of values from neighbors
  \end{itemize}

  \textbf{Handling Circular Dependencies}
  \begin{itemize}
    \item \textbf{Chaotic Iteration}: Keep applying transfer function until no value at any node can change
    \item \textbf{Work-List Iteration}: Keep a list of nodes to evaluate. If a value changes in some node, add its neighbors to the work list.
    \item \textbf{Interval Analysis}:
  \end{itemize}
  
  \textbf{Common Data Flow Facts of Interest (Questions)}
  \begin{itemize}
    \item Is variable $x$ live at this statement?
    \item Is $expr_i$ available at this statement?
    \item Is $expr_i$ invariable in the loop?
    \item Can constant (copy) propagation possible?
  \end{itemize}

\subsection{Abstract Interpretation}


\section{Garbage Collection}

  \textbf{Definitions}:
  
  \begin{itemize}
    \item \textit{Object Space}: Area of the heap where declared objects are stored.
  \end{itemize}

  \textbf{Mental Model} -- Rooted Graph
  \begin{enumerate}
    \item \textit{Vertices}: Objects on the stack and heap
    \item \textit{Directed Edges}: Pointer from stack object to a heap object \textbf{or} from one heap object to another
    \item \textit{Roots}: Objects on the stack or in registers
    \item \textit{Garbage}: Any object on the heap that is \textbf{unreachable from a root}
  \end{enumerate}

  \textbf{Motivations for Garbage Collection}
  \begin{enumerate}
    \item Programmers make mistakes and forget to free memory
    \item Programmers write messier, less efficient code trying to avoid mistakes
    \item Storage management can be complicated
    \item Garbage collection can lead to cleaner, more modular code with fewer bugs
  \end{enumerate}

\subsection*{Mark and Sweep}
  \textit{Phase \#1}: \textbf{Mark} -- Recursively traverse (with a stack) from the roots marking any encountered object ``reachable''
  
  \textit{Phase \#2}: \textbf{Sweep} -- Sequentially sweep through object space reclaiming anything that is not marked.
  
  \textbf{Disadvantages:}
  \begin{enumerate}
    \item May lead to poor cache locality (i.e. fragmentation)
    \item Requires \textit{stop the world} to perform collection
  \end{enumerate}
  
\subsection*{Other Ideas}

  \textbf{Generational Collecting}: Multiple ``generations'' of objects depending on when the object was allocated. Transforms complexity from proportional to heap size to proportional to generation size. \textit{Intuition}: Most objects are discarded shortly after being used.
  \begin{itemize}
    \item Complex to implement, in particular if concurrent
    \item May not be \textit{comprehensive}, i.e., may not successfully remove all (long-lived) garbage.
  \end{itemize}

  \textbf{Reference Counting}: Store a counter with each object denoting the number of references to the object.  Delete the object when the reference count equals zero.
  \begin{itemize}
    \item Sensitive to circular structures, i.e., $A \rightarrow B$ and $A \leftarrow B$.
    \item Adds overhead
  \end{itemize}

  \textbf{Concurrency}: Implement garbage collection in a separate thread to eliminate ``stop the world''
  
  \textbf{Tuning}: Different collection strategies may be superior depending on the language or language-type (e.g., functional)

\end{document}